\documentclass[journal,12pt,twocolumn]{IEEEtran}
\usepackage{amsmath}
\usepackage{amsthm}
    \usepackage[latin1]{inputenc}                                 %%
    \usepackage{color}                                            %%
    \usepackage{array}                                            %%
    \usepackage{longtable}                                        %%
    \usepackage{calc}                                             %%
    \usepackage{multirow}                                         %%
    \usepackage{hhline}                                           %%
    \usepackage{ifthen}                                           %%
  %optionally (for landscape tables embedded in another document): %%
    \usepackage{lscape}     
%\usepackage{iithtlc}
\usepackage{circuitikz}
\usepackage{tikz}
\usetikzlibrary{arrows, shapes.gates.logic.US, calc}

\def\inputGnumericTable{}                                 %%
\begin{document}
\theoremstyle{definition}
\newtheorem{theorem}{Theorem}[section]
\newtheorem{problem}{Problem}
\newcommand{\solution}{\noindent \textbf{Solution: }}
\title{
%\logo{
Template for Logic Gates
%}
}
\author{Parijat Mitra$^{\dagger}$ and GVV Sharma$^{*}$
\thanks{$\dagger$ Parijat is a UG student at IIT Bhilai. He did this work as an intern at IIT Hyderabad. *The author is with the Department
of Electrical Engineering, Indian Institute of Technology, Hyderabad
502285 India e-mail:  gadepall@iith.ac.in. All content in this manual is released under GNU GPL.  Free and open source.}
}
\maketitle
\begin{abstract}
This manual shows how to draw logic gates.
\end{abstract}

\begin{figure}[!h]
\centering
\resizebox {\columnwidth} {!} {
\begin{tikzpicture}
    \node (x) at (0.5, 2) {$\bar{X}$};
    \node (y) at (0, 1) {$Y$};
    
    
    \node[nand gate US, draw, rotate=0, logic gate inputs=nn] at ($(y) + (2, 0.085)$) (xory) {};
     \draw (x) -- ([xshift=0.2cm]x) |- (xory.input 1);
    \draw (y) -- ([xshift=0.2cm]y) |-(xory.input 2);
    
    \node (x) at (0.5, -1) {$\bar{W}$};
    \node (y) at (0, -2) {$Y$};
    
    
    \node[nand gate US, draw, rotate=0, logic gate inputs=nn] at ($(y) + (2, 0.085)$) (xory1) {};
    \draw (x) -- ([xshift=0.2cm]x) |-(xory1.input 1);
    \draw (y) -- ([xshift=1cm]y) |-(xory1.input 2);
    
      
    \node (x) at (0.5, -4) {$W$};
    \node (y) at (0, -5) {$X$};
    \node (z) at (0.7, -6) {$\bar{Y}$};

    
    \node[nand gate US, draw, rotate=0, logic gate inputs=nnn] at ($(y) + (2, 0.085)$) (xory3) {};
    \draw (x) -- ([xshift=0.2cm]x) |-(xory3.input 1);
    \draw (y) -- ([xshift=0.4cm]y) |-(xory3.input 2);
    \draw (z) -| ($(y) + (0.4, -0.3)$) |- (xory3.input 3);
   
    
    \node[nand gate US, draw, rotate=0, logic gate inputs=nnn] at ($(xory1.output) + (2, 0)$) (xory4) {};

    \draw (xory.output) -- ([xshift=0.2cm]xory.output) |-(xory4.input 1);
    \draw (xory1.output) -- ([xshift=0.2cm]xory1.output) |-(xory4.input 2);
    \draw (xory3.output) -- ([xshift=0.2cm]xory3.output) |-(xory4.input 3);

     \draw (xory4.output) -- node[above]{$\bar{X}Y+\bar{W}Y+WX\bar{Y}$} ($(xory4) + (4, 0)$);
\end{tikzpicture}

}
\caption{CMOS Logic for $\overline{C}$}
\label{fig:CMOS_C}
\end{figure}
\end{document}




