\renewcommand{\theequation}{\theenumi}
\renewcommand{\thefigure}{\theenumi}
\begin{enumerate}[label=\thesubsection.\arabic*.,ref=\thesubsection.\theenumi]
\numberwithin{equation}{enumi}
\numberwithin{figure}{enumi}
\numberwithin{table}{enumi}

\item आकृति \ref{fig:inc_kmapX_A} के द्वारा $A$ के व्यंजक को व्युत्पन्न करें।

\solution

\begin{align}
\label{eq:kmapX_disp_A}
A &= W^{\prime}
\end{align}
%

\begin{figure}[!ht]
\centering
\resizebox{\columnwidth}{!} {
\begin{karnaugh-map}[4][4][1][][]
    \maxterms{1,3,5,7,9}
    \terms{10,11,12,13,14,15}{X}
    \minterms{0,2,4,6,8}
%    \implicantedge{0}{4}{2}{6}
  %  \implicantedge{0}{0}{8}{8}
  \implicantedge{0}{8}{2}{10}
    % note: posistion for start of \draw is (0, Y) where Y is
    % the Y size(number of cells high) in this case Y=2
    \draw[color=black, ultra thin] (0, 4) --
    node [pos=0.7, above right, anchor=south west] {$XW$} % Y label
    node [pos=0.7, below left, anchor=north east] {$ZY$} % X label
    ++(135:1);
        
    \end{karnaugh-map}

}
\caption{$A$ का निर्गुण विवक्षक कृत क-मानचित्र।}
\label{fig:inc_kmapX_A}
\end{figure}
%

\item आकृति \ref{fig:inc_kmapX_B} के द्वारा $B$ के व्यंजक को व्युत्पन्न करें।

\solution

\begin{align}
\label{eq:kmapX_disp_B}
B &= WX^{\prime}Z^{\prime}+W^{\prime}X
\end{align}
%

\begin{figure}[!ht]
\centering
\resizebox{\columnwidth}{!} {
\begin{karnaugh-map}[4][4][1][][]
    \minterms{1,2,5,6}
    \maxterms{0,3,4,7,8,9,10,11,12,13,14,15}
    \implicant{12}{10}
    \implicant{3}{11}
    \implicant{0}{8}
    % note: position for start of \draw is (0, Y) where Y is
    % the Y size(number of cells high) in this case Y=2
    \draw[color=black, ultra thin] (0,4) --
    node [pos=0.7, above right, anchor=south west] {$XW$} % YOU CAN CHANGE NAME OF VAR HERE, THE $X$ IS USED FOR ITALICS
    node [pos=0.7, below left, anchor=north east] {$ZY$} % SAME FOR THIS
    ++(135:1);
        
    \end{karnaugh-map}

}
\caption{$B$ का निर्गुण विवक्षक कृत क-मानचित्र।}
\label{fig:inc_kmapX_B}
\end{figure}
%
\item आकृति \ref{fig:inc_kmapX_C} के द्वारा $C$ के व्यंजक को व्युत्पन्न करें।

\solution

\begin{align}
\label{eq:kmapX_disp_C}
C &= X^{\prime}Y+W^{\prime}Y++WXY^{\prime}
\end{align}
%

\begin{figure}[!ht]
\centering
\resizebox{\columnwidth}{!} {
\begin{karnaugh-map}[4][4][1][][]
    \minterms{3,4,5,6}
    \maxterms{0,1,2,7,8,9,10,11,12,13,14,15}
    \implicant{12}{10}
    \implicant{0}{1}
    \implicant{15}{7}
    \implicantcorner
    
    % note: position for start of \draw is (0, Y) where Y is
    % the Y size(number of cells high) in this case Y=2
    \draw[color=black, ultra thin] (0, 4) --
    node [pos=0.7, above right, anchor=south west] {$XW$} % YOU CAN CHANGE NAME OF VAR HERE, THE $X$ IS USED FOR ITALICS
    node [pos=0.7, below left, anchor=north east] {$ZY$} % SAME FOR THIS
    ++(135:1);
        
    \end{karnaugh-map}

}
\caption{$C$ का निर्गुण विवक्षक कृत क-मानचित्र।}
\label{fig:inc_kmapX_C}
\end{figure}
%
%
\item आकृति \ref{fig:inc_kmapX_D} के द्वारा $D$ के व्यंजक को व्युत्पन्न करें।

\solution

\begin{align}
\label{eq:kmapX_disp_D}
D &= WXY+W^{\prime}Z
\end{align}
%

\begin{figure}[!ht]
\centering
\resizebox{\columnwidth}{!} {
\begin{karnaugh-map}[4][4][1][][]
    \maxterms{0,1,2,3,4,5,6,9}
    \terms{10,11,12,13,14,15}{X}
    \minterms{7,8}
     \implicant{7}{15}    
    \implicantedge{12}{8}{14}{10}
%     \implicantedge{8}{10}{12}{14}
    % note: posistion for start of \draw is (0, Y) where Y is
    % the Y size(number of cells high) in this case Y=2
    \draw[color=black, ultra thin] (0, 4) --
    node [pos=0.7, above right, anchor=south west] {$XW$} % Y label
    node [pos=0.7, below left, anchor=north east] {$ZY$} % X label
    ++(135:1);
        
    \end{karnaugh-map}

}
\caption{$D$ का निर्गुण विवक्षक कृत क-मानचित्र।}
\label{fig:inc_kmapX_D}
\end{figure}
%

\end{enumerate}
