%\addcontentsline{toc}{chapter}{Nomenclature}

\nomenclature{Seven Segment Display}{ सप्तांश प्रदर्शी} 
\nomenclature{Code}{ गूढ़} 
\nomenclature{Decoder}{ गूढ़वाचक}
\nomenclature{Incrementing}{ परवर्ती}
\nomenclature{Karnaugh Map}{ कार्नो मानचित्र}
\nomenclature{Implicant}{ विवक्षक}
\nomenclature{LED}{ प्रकाश उत्सर्जक यंत्र}
\nomenclature{Table}{ सारणी}
\nomenclature{Figure}{ आकृति}
\nomenclature{Variable}{ चर}
\nomenclature{Input}{ आगत}
\nomenclature{Output}{ निर्गत}
\nomenclature{Boolean Algebra}{ बूलीय बीजगणित}
\nomenclature{Verify}{ सत्यापित}
\nomenclature{Combinational Logic}{ संयोजक तर्क}
\nomenclature{Minimize}{ कनिष्ठीकरण}
\nomenclature{Execute}{ निष्पादित}
\nomenclature{Expression}{ व्यंजक}
\nomenclature{Equation}{ समीकरण}
\nomenclature{Axiom}{ अभिगृह}
\nomenclature{Reduced}{ समानयनिक}
\nomenclature{C Program}{ C क्रमादेश}
\nomenclature{Binary}{ द्विआधारी}
\nomenclature{Truth Table}{सत्य सारणी}
\nomenclature{Derive}{व्युत्पन्न }
\nomenclature{XOR}{अर्ध योग}
\nomenclature{Complement}{पूरक}
\nomenclature{Decade Counter}{दशक गणित्र}
\nomenclature{Digital Logic}{अंकीय तर्क}
\nomenclature{Function}{फलन}



 
