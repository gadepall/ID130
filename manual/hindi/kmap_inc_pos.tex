
%\renewcommand{\theequation}{\thesection}
\begin{enumerate}[label=\thesection.\arabic*.,ref=\thesection.\theenumi]
%\begin{enumerate}[label=\thesubsection.\arabic*.,ref=\thesubsection.\theenumi]
\numberwithin{equation}{enumi}
\numberwithin{figure}{enumi}
\numberwithin{figure}{section}

%\setcounter{figure}{0}
%\numberwithin{table}{section}
%\numberwithin{equation}{section}
%\renewcommand{\thefigure}{\thesection}
%\numberwithin{figure}{section}
\item सारणी  \ref{table:counter_decoder} में 0 संख्याओं  की  क-मानचित्र. \ref{fig:kmap_A_pos}-\ref{fig:kmap_D_pos}  में संयुक्ति से निम्न योग-गुणन व्यंजक उपलब्ध होते हैं ।  पूर्व खंडों में समस्त बूलीय फलन के-मानचित्र में 1 अंकों के संयोग  से फलन गुणन-योग रूप में प्राप्त हुए थे.
%
\begin{align}
\label{eq:kmap_A_pos}
A&=(Z^{\prime}+Y^{\prime})W^{\prime}(Z^{\prime}+X^{\prime})
\\
\label{eq:kmap_B_pos}
    B&=(X^{\prime}+W^{\prime})Z^{\prime}(X+W)
\\
\label{eq:kmap_C_pos}
     C&=(Z+Y+X)(Y^{\prime}+X^{\prime}+W^{\prime})(X^{\prime}+Y+W)Z^{\prime}
\\
     D&=(Z+Y)(Y^{\prime}+X)(X+W^{\prime})(X^{\prime}+W)(Z^{\prime}+X^{\prime})
\label{eq:kmap_D_pos}
\end{align}
%



\begin{figure}[!h]
\centering
\resizebox{\columnwidth}{!}
{
\begin{karnaugh-map}[4][4][1][][]
    \maxterms{1,3,5,7,9,10,11,12, 13 ,14,15}
    \minterms{0,2,4,8}
    \implicant{12}{14}
    \implicant{1}{11}
    \implicant{15}{10}
    % note: position for start of \draw is (0, Y) where Y is
    % the Y size(number of cells high) in this case Y=2
    \draw[color=black, ultra thin] (0, 4) --
    node [pos=0.7, above right, anchor=south west] {$XW$} % YOU CAN CHANGE NAME OF VAR HERE, THE $X$ IS USED FOR ITALICS
    node [pos=0.7, below left, anchor=north east] {$ZY$} % SAME FOR THIS
    ++(135:1);
        
    \end{karnaugh-map}
}
\caption{$A$ का योग-गुणन}
\label{fig:kmap_A_pos}
\end{figure}


\begin{figure}[!h]
\centering
\resizebox{\columnwidth}{!}
{
\begin{karnaugh-map}[4][4][1][][]
    \minterms{1,2,5,6}
    \maxterms{0,3,4,7,8,9,10,11,12,13,14,15}
    \implicant{12}{10}
    \implicant{3}{11}
    \implicant{0}{8}
    % note: position for start of \draw is (0, Y) where Y is
    % the Y size(number of cells high) in this case Y=2
    \draw[color=black, ultra thin] (0,4) --
    node [pos=0.7, above right, anchor=south west] {$XW$} % YOU CAN CHANGE NAME OF VAR HERE, THE $X$ IS USED FOR ITALICS
    node [pos=0.7, below left, anchor=north east] {$ZY$} % SAME FOR THIS
    ++(135:1);
        
    \end{karnaugh-map}
}
\caption{$B$ का योग-गुणन}
\label{fig:kmap_B_pos}
\end{figure}
    

\begin{figure}[!h]
\centering
\resizebox{\columnwidth}{!}
{
\begin{karnaugh-map}[4][4][1][][]
    \minterms{3,4,5,6}
    \maxterms{0,1,2,7,8,9,10,11,12,13,14,15}
    \implicant{12}{10}
    \implicant{0}{1}
    \implicant{15}{7}
    \implicantcorner
    
    % note: position for start of \draw is (0, Y) where Y is
    % the Y size(number of cells high) in this case Y=2
    \draw[color=black, ultra thin] (0, 4) --
    node [pos=0.7, above right, anchor=south west] {$XW$} % YOU CAN CHANGE NAME OF VAR HERE, THE $X$ IS USED FOR ITALICS
    node [pos=0.7, below left, anchor=north east] {$ZY$} % SAME FOR THIS
    ++(135:1);
        
    \end{karnaugh-map}
}
\caption{$C$ का योग-गुणन}
\label{fig:kmap_C_pos}
\end{figure}

     
\begin{figure}[!h]
\centering
\resizebox{\columnwidth}{!}
{
\begin{karnaugh-map}[4][4][1][][]
    \maxterms{0,1,2,3,4,5,6,9,10,11,12,13,14,15}
    \minterms{7,8}
    \implicant{15}{10}
    \implicant{2}{10}
    \implicant{0}{2}
    \implicant{4}{13}
    \implicant{1}{9}
    % note: position for start of \draw is (0, Y) where Y is
    % the Y size(number of cells high) in this case Y=2
    \draw[color=black, ultra thin] (0, 4) --
    node [pos=0.7, above right, anchor=south west] {$XW$} % YOU CAN CHANGE NAME OF VAR HERE, THE $X$ IS USED FOR ITALICS
    node [pos=0.7, below left, anchor=north east] {$ZY$} % SAME FOR THIS
    ++(135:1);
        
    \end{karnaugh-map}
}
\caption{$D$ का योग-गुणन}
\label{fig:kmap_D_pos}
\end{figure}

\item  सारणी  \ref{table:disp_dec} में 0 संख्याओं  की  क-मानचित्र में संयुक्ति से योग-गुणन व्यंजक उपलब्ध करें।
\end{enumerate}
