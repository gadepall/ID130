निम्न  चरणों में सारणी \ref{table:disp_dec}  एवं  निर्गुण  विवक्षकृत  क-मानचित्र के द्वारा $a,b,c,d,e,f,g$ के न्यूनतम व्यंजक को व्युत्पन्न किया जाएगा.

\renewcommand{\theequation}{\theenumi}
\renewcommand{\thefigure}{\theenumi}
\begin{enumerate}[label=\thesubsection.\arabic*.,ref=\thesubsection.\theenumi]
\numberwithin{equation}{enumi}
\numberwithin{figure}{enumi}
\numberwithin{table}{enumi}

\item आकृति \ref{fig:disp_kmapX_a} के द्वारा $a$ के व्यंजक को व्युत्पन्न करें.


\solution

\begin{align}
\label{eq:kmapX_disp_a}
a &= A B^{\prime}C^{\prime} D^{\prime} + A^{\prime}B^{\prime} C 
\end{align}
%
\begin{figure}[!ht]
\centering
\resizebox{\columnwidth}{!} {

\begin{karnaugh-map}[4][4][1][][]
    \maxterms{0,2,3,5,6,7,8,9}

    \minterms{1,4}
%    \implicantedge{1}{1}{9}{9}
    \implicant{4}{4}
    \implicant{1}{1}
    
    % note: position for start of \draw is (0, Y) where Y is
    % the Y size(number of cells high) in this case Y=2
    \draw[color=black, ultra thin] (0, 4) --
    node [pos=0.7, above right, anchor=south west] {$BA$} % YOU CAN CHANGE NAME OF VAR HERE, THE $X$ IS USED FOR ITALICS
    node [pos=0.7, below left, anchor=north east] {$DC$} % SAME FOR THIS
    ++(135:1);
        
    \end{karnaugh-map}

}
\caption{$a$ का निर्गुण  विवक्षकृत क-मानचित्र.}
\label{fig:disp_kmapX_a}
\end{figure}
%

\item आकृति \ref{fig:disp_kmapX_b} के द्वारा $b$ के व्यंजक को व्युत्पन्न करें.


\solution

\begin{align}
\label{eq:kmapX_disp_b}
b &= A B^{\prime}C  + A^{\prime}B C 
\\
&= C (A\oplus B)
\end{align}
%


\begin{figure}[!ht]
\centering
\resizebox{\columnwidth}{!} {
\begin{karnaugh-map}[4][4][1][][]
    \maxterms{5,6}

    \indeterminants{10,11,12,13,14,15}
%    \dontcare{10,11,12,13,14,15}
    \minterms{0,1,2,3,4,7,8,9}
    \implicant{0}{8}
    \implicant{3}{11}
    \implicantedge{0}{2}{8}{10}
    % note: position for start of \draw is (0, Y) where Y is
    % the Y size(number of cells high) in this case Y=2
    \draw[color=black, ultra thin] (0, 4) --
    node [pos=0.7, above right, anchor=south west] {$XW$} % YOU CAN CHANGE NAME OF VAR HERE, THE $X$ IS USED FOR ITALICS
    node [pos=0.7, below left, anchor=north east] {$ZY$} % SAME FOR THIS
    ++(135:1);
        
    \end{karnaugh-map}

}
\caption{$b$ का निर्गुण  विवक्षकृत क-मानचित्र।}
\label{fig:disp_kmapX_b}
\end{figure}
%
\item आकृति \ref{fig:disp_kmapX_c} के द्वारा $c$ के व्यंजक को व्युत्पन्न करें।


\solution

\begin{align}
\label{eq:kmapX_disp_c}
c &=  A^{\prime}B C^{\prime}
\end{align}
%
\begin{figure}[!ht]
\centering
\resizebox{\columnwidth}{!} {

\begin{karnaugh-map}[4][4][1][][]
    \maxterms{1,4,0,3,5,6,7,8,9}
%    \terms{}{X}
%    \terms{10,11,12,13,14,15}{X}
    \minterms{2}
%    \implicantedge{1}{1}{9}{9}
%    \implicant{4}{4}
    \implicant{2}{2}
    
    % note: position for start of \draw is (0, Y) where Y is
    % the Y size(number of cells high) in this case Y=2
    \draw[color=black, ultra thin] (0, 4) --
    node [pos=0.7, above right, anchor=south west] {$BA$} % YOU CAN CHANGE NAME OF VAR HERE, THE $X$ IS USED FOR ITALICS
    node [pos=0.7, below left, anchor=north east] {$DC$} % SAME FOR THIS
    ++(135:1);
        
    \end{karnaugh-map}

}
\caption{$c$ का निर्गुण  विवक्षकृत क-मानचित्र।}
\label{fig:disp_kmapX_c}
\end{figure}
%
\item  आकृति \ref{fig:disp_kmapX_d} के द्वारा $d$ के व्यंजक को व्युत्पन्न करें।
\\
\solution


\begin{align}
\label{eq:kmapX_disp_d}
d=AB^{\prime}C^{\prime}+A^{\prime}B^{\prime}C+ABC
\end{align}
%
\begin{figure}[!ht]
\centering
\resizebox{\columnwidth}{!} {

\begin{karnaugh-map}[4][4][1][][]
    \maxterms{0,2,3,5,6,8}
    \minterms{1,4,7,9}
    \implicantedge{1}{1}{9}{9}
    \implicant{4}{4}
    \implicant{7}{7}
    
    % note: position for start of \draw is (0, Y) where Y is
    % the Y size(number of cells high) in this case Y=2
    \draw[color=black, ultra thin] (0, 4) --
    node [pos=0.7, above right, anchor=south west] {$BA$} % YOU CAN CHANGE NAME OF VAR HERE, THE $X$ IS USED FOR ITALICS
    node [pos=0.7, below left, anchor=north east] {$DC$} % SAME FOR THIS
    ++(135:1);
        
    \end{karnaugh-map}

}
\caption{$d$ का निर्गुण  विवक्षकृत क-मानचित्र।}
\label{fig:disp_kmapX_d}
\end{figure}
\item आकृति \ref{fig:disp_kmapX_e} के द्वारा $e$ के व्यंजक को व्युत्पन्न करें।
%

\begin{align}
\label{eq:kmapX_disp_e}
e=A+B^{\prime}C
\end{align}
%
\begin{figure}[!ht]
\centering
\resizebox{\columnwidth}{!} {
\begin{karnaugh-map}[4][4][1][][]
    \maxterms{0,1,3,4,5,6,7,11,13,15}
    \minterms{2,8,9,10,12,14}
    
    \implicantedge{12}{8}{14}{10}
    \implicantedge{2}{2}{10}{10}
    \implicant{8}{9}

    % note: position for start of \draw is (0, Y) where Y is
    % the Y size(number of cells high) in this case Y=2
    \draw[color=black, ultra thin] (0, 4) --
    node [pos=0.7, above right, anchor=south west] {$CD$} % YOU CAN CHANGE NAME OF VAR HERE, THE $X$ IS USED FOR ITALICS
    node [pos=0.7, below left, anchor=north east] {$AB$} % SAME FOR THIS
    ++(135:1);
        
    \end{karnaugh-map}


}
\caption{$e$ का निर्गुण  विवक्षकृत क-मानचित्र।}
\label{fig:disp_kmapX_e}
\end{figure}
%
\item आकृति \ref{fig:disp_kmapX_f} के द्वारा $f$ के व्यंजक को व्युत्पन्न करें।
%
\begin{align}
\label{eq:kmapX_disp_f}
f= AB + AC^{\prime}D^{\prime} + BC^{\prime}
\end{align}
%
\begin{figure}[!ht]
\centering
\resizebox{\columnwidth}{!} {
\begin{karnaugh-map}[4][4][1][][]
    \maxterms{0,1,2,3,5,6,7,10,9,13,14,15}
    \minterms{4,8,12,11}
    \implicant{11}{11}
    \implicant{4}{12}
    \implicant{8}{8}
    % note: position for start of \draw is (0, Y) where Y is
    % the Y size(number of cells high) in this case Y=2
    \draw[color=black, ultra thin] (0, 4) --
    node [pos=0.7, above right, anchor=south west] {$CD$} % YOU CAN CHANGE NAME OF VAR HERE, THE $X$ IS USED FOR ITALICS
    node [pos=0.7, below left, anchor=north east] {$AB$} % SAME FOR THIS
    ++(135:1);
        
    \end{karnaugh-map}

}
\caption{$f$ का निर्गुण  विवक्षकृत क-मानचित्र।}
\label{fig:disp_kmapX_f}
\end{figure}

%
\item आकृति \ref{fig:disp_kmapX_g} के द्वारा $g$ के व्यंजक को व्युत्पन्न करें।
%
\begin{align}
\label{eq:kmapX_disp_g}
g = B^{\prime}C^{\prime}D^{\prime}+ABC
\end{align}
%
\begin{figure}[!ht]
\centering
\resizebox{\columnwidth}{!} {
\begin{karnaugh-map}[4][4][1][][]
    \maxterms{2,3,4,5,6,8,9}
    \minterms{0,1,7}

    \implicant{0}{1}
    \implicant{7}{7}
    
    % note: position for start of \draw is (0, Y) where Y is
    % the Y size(number of cells high) in this case Y=2
    \draw[color=black, ultra thin] (0, 4) --
    node [pos=0.7, above right, anchor=south west] {$BA$} % YOU CAN CHANGE NAME OF VAR HERE, THE $X$ IS USED FOR ITALICS
    node [pos=0.7, below left, anchor=north east] {$DC$} % SAME FOR THIS
    ++(135:1);
  
      
    \end{karnaugh-map}

}
\caption{$g$ का निर्गुण  विवक्षकृत क-मानचित्र।}
\label{fig:disp_kmapX_g}
\end{figure}
\end{enumerate}
%
%
